\documentclass{beamer}

% Set up
\usepackage{xcolor}
\usepackage{graphicx}
\usepackage{amsmath}
\usepackage{graphicx}
\usepackage{amsthm}
\usepackage{mathrsfs}  
\usepackage{amsfonts}
\usepackage{amssymb}
\usepackage{amsmath}
% \usepackage[utf8]{inputenc}

% Beamer setup
\usetheme{default}
\usecolortheme{default}
% \AtBeginSection[]
% {
%   \begin{frame}
%     \frametitle{Table of Contents}
%     \tableofcontents[currentsection]
%   \end{frame}
% }
\usefonttheme[onlymath]{serif}

% Math commands
\newcommand{\1}{\mathbbm{1}}
% measures / sets of measures:
\newcommand{\mH}{\mathscr{H}}
\newcommand{\mL}{\mathscr{L}}
\newcommand{\mP}{\mathscr{P}}
\newcommand{\cH}{\mathcal{H}} % entropy
\newcommand{\cT}{\mathcal{T}} % topology
\newcommand{\cB}{\mathcal{B}} % borel
\newcommand{\W}{\mathcal{W}} % new metric W
\newcommand{\X}{\mathcal{X}} % space X
\newcommand{\nabladot}{\nabla\cdot} % discrete divergence operator
\newcommand{\A}{\mathcal{A}} % action
\newcommand*\de{\mathop{}\!\mathrm{d}}
\DeclareMathOperator*{\argmin}{arg\,min}
\DeclareMathOperator*{\argmax}{arg\,max}
\DeclareMathOperator*{\supp}{supp}
\DeclareMathOperator*{\id}{id}
\DeclareMathOperator{\imm}{Im}
\DeclareMathOperator{\lalpha}{\overline{\alpha}}
\DeclareMathOperator{\aconv}{\alpha \textit{x}+\overline{\alpha}\textit{y}}
\DeclareMathOperator{\bconv}{\beta \textit{x}+\overline{\beta}\textit{y}}

%%%%%%%%%%%%%%%%%%%% from the paper
\newcommand{\R}{\ensuremath{\mathbb{R}}}
\newcommand{\Z}{\ensuremath{\mathbb{Z}}}
\DeclareMathOperator{\Ric}{Ric}
\DeclareMathOperator{\vol}{vol}

\def\diam{\mathop{\mathrm{diam}}\nolimits}
\def\vol{\mathop{\mathrm{vol}}\nolimits}
\def\area{\mathop{\mathrm{area}}\nolimits}
\def\Hess{\mathop{\mathrm{Hess}}\nolimits}
\def\supp{\mathop{\mathrm{supp}}\nolimits}
\def\Ric{\mathop{\mathrm{Ric}}\nolimits}
\def\Id{\mathop{\mathrm{Id}}\nolimits}
\def\tr{\mathop{\mathrm{tr}}\nolimits}
\def\ac{\mathop{\mathrm{ac}}\nolimits}
\def\div{\mathop{\mathrm{div}}\nolimits}
%%%%%%%%%%%%%%%%%%%%%%%%%%%%%%%%%%%%



%Information to be included in the title page:
\title[Inequalities for Markov chains with non-neg Ricci]{Poincaré, modified logarithmic Sobolev and
isoperimetric inequalities for Markov chains with
non-negative Ricci curvature \\~\\ \small M. Erbar and M. Fathi} 
\author[Matilde Dolfato]{\small Matilde Dolfato \\ Mentor: Prof. Elia Brué}
\institute[]{Università Bocconi  \par Visiting Student Initiative - BIDSA}
\date{\today}

\begin{document}

\frame{\titlepage}
% % comment on title

\begin{frame}{Overview of the talk}
    \tableofcontents
\end{frame}

\section{Main result: convergence to equilibrium of the zero-range process}

\subsection{The zero-range process}
\begin{frame}[allowframebreaks]{The zero-range process}
    Consider $K$ interacting particles on the complete graph with $L$ sites.\\~\\

    The state space is $\X_{K,L}=\{\eta \in \mathbb{N}^L: \sum_i \eta_i = K\}$. \\~\\

    The zero-range process with \textbf{constant rates} can be described as:
    \begin{enumerate}
        \item Choose a site $i$ uniformly at random
        \begin{enumerate}
            \item If $\eta_i=0$, do nothing
            \item Else, choose another site $j$ uniformly at random and move 1 particle from $i$ to $j$
        \end{enumerate}
    \end{enumerate}
    $\eta^{i,j}$ denotes the new configuration after such a move.

    \begin{equation*}
    Q_{K,L}(\eta,\theta)=
    \begin{cases}
        \frac{1}{L} & \theta=\eta^{i,j} \text{ for some }i, j\\
        0 & \text{else}
    \end{cases}
    \end{equation*}
    Invariant measure is the uniform measure $\pi_{K,L}$.
\\~\\
Good model for:
\begin{itemize}
    \item traffic flow
    \item population dynamics
    \item particle physics
    \item ... any system in which deciding to move depends on the surroundings
\end{itemize}
\end{frame}

\subsection{Bound on mixing time and interpretation}
\begin{frame}{Main result and interpretation}
    % HIGH LEVEL dont even tell them it's called log Sobolev or just mention
    \begin{block}{Convergence to equilibrium}
        The constant-rate zero range process with $K$ particles and $L$ sites has mixing time bounded by
        \begin{equation}
        \tau_{mix}(\varepsilon) \le K L \log L \bigg(\frac{1}{8} - \frac{\log \varepsilon}{c}\bigg)
        \end{equation}
        for some universal constant $c$. 
    \end{block}
\pause
\begin{center}\textcolor{blue}{$\star$ This is convergence to equilibrium! $\star$}\end{center}
\end{frame}


\section{From geometry to inequalities}
\tableofcontents[currentsection]
\subsection{Setup}
\begin{frame}{Setup}
    Finite space of states $\X$, irreducible, reversible Markov kernel $Q$ and stationary measure $\pi$
  \[
   Q(x,y)\pi(x)=Q(y,x)\pi(y)
  \]
  Operator $L$ acting on functions $\psi:\X \to\R$ is\[L\psi(x)=\sum_{y\in\X}(\psi(y)-\psi(x))Q(x,y)\]
  is the generator of a continuous time Markov chain.

 Space of probability densities
  \[
    \mP(\X)
    := \Bigl\{\rho:\X\to\R_+ : \sum_{x\in\X} \rho(x)\,\pi(x) = 1\Bigr\}
  \]
\end{frame}
\begin{frame}{Distance $\W$}
    We define a discrete transport distance, by analogy with the Wasserstein distance. For $\rho_0,\rho_1 \in \mP(\X),$
    \[
\W(\rho_0, \rho_1)^2 := 
\inf_{\rho, \psi} 
\; \frac{1}{2} 
\int_0^1 \sum_{x,y \in X} 
\bigl( \psi_t(x) - \psi_t(y) \bigr)^2 
\, \hat{\rho}_t(x,y) \, Q(x,y) \, \pi(x) 
\, dt.
\]
where the infimum runs over all sufficiently regular curves satisfying a continuity equation
\[
\begin{cases}
\displaystyle
\frac{d}{dt}\rho_t(x) 
+ \sum_{y \in \mathcal{X}} 
\bigl( \psi_t(y) - \psi_t(x) \bigr)
\hat{\rho}_t(x,y) Q(x,y) 
= 0, 
& \forall x \in \mathcal{X}, \\[1.2em]
\rho|_{t=0} = \rho_0, \quad 
\rho|_{t=1} = \rho_1.
\end{cases}
\]
and $\hat\rho(x,y)=\theta(\rho(x),\rho(y))$ which is the logarithmic mean.
\\~\\

Endowing $\X$ with this distance, we obtain that the Markov semigroup $P_t=e^{tL}$ is the gradient flow of Shannon's entropy $\mathcal{H}(\rho)=\sum_{x\in\X} \rho(x) \log \rho(x) \pi(x)$
\end{frame}
\subsection{Ricci for discrete processes}

\begin{frame}{Ricci for discrete processes}
    \begin{block}{Entropic Ricci curvature for discrete space}
    $(\X,Q,\pi)$ has entropic Ricci curvature bounded from below by $\kappa \in \R$ if for any $\rho_t \in (\mP(\X), \W)$ we have
    \[
    \mathcal{H}(\rho_t) \le (1-t)\mathcal{H}(\rho_0) + t\mathcal{H}(\rho_1)- \frac{\kappa}{2}t(1-t)\W(\rho_0,\rho_1)^2
    \]
    In this case, we write $\Ric(\X,Q,\pi)\ge \kappa$.
    \end{block}
\end{frame}
\begin{frame}{Ricci for the zero-range process}
    In another contribution, M. Fathi and J. Maas provide explicit ways of computing bounds on Ricci for discrete processes
    \\~\\
    They show that for the ZRP with \textit{increasing} rates, Ricci is positive
    \\~\\
    The ZRP with \textit{constant} rates has $\Ric \ge 0$
\end{frame}

\subsection{Modified Logarithmic Sobolev inequality for discrete processes with non-neg Ricci}
\begin{frame}[allowframebreaks]{The Modified log-Sobolev inequality}
\begin{block}{Modified logarithmic Sobolev inequality}
Bound on the entropy in terms of a norm of the gradient of our observable:
    \begin{equation*}\tag{MLSI}
        \mathcal{H}(\rho) \le \frac{1}{2\lambda} \Big\lVert \nabla \log\rho \Big\rVert^2 = \frac{1}{2\lambda} \mathcal{I}(\rho)
    \end{equation*}
where $\mathcal{I}$ is the Fisher information.
\end{block}

Implications:
\begin{itemize}
    \item encodes the distribution of the eigenvalues of the Laplacian (generator)
    \item entropy decay along the heat flow \(\mathcal{H}(P_t \rho) \le e^{-2\lambda t}\mathcal{H}(\rho)\)
    \item with Poincaré ineq,, it implies \textcolor{blue}{\textbf{convergence to equilibrium}}
    \item convergence in Wassertsein distance \(\W(\rho_t, \sigma_t) \le e^{-2\lambda t} \W(\rho_0, \sigma_0)\)
    \item hypercontractivity \(\lVert P_t \rho\rVert\le \lVert\rho \rVert\)
\end{itemize}

We look at convergence to equilibrium in terms of the \textbf{mixing time} of the Markov chain. 

\begin{block}{Total variation mixing time of a Markov chain}
    For $\varepsilon > 0$, 
    \[
    \tau_{mix}(\varepsilon) := \inf \Big\{t>0 : \lVert P_t^* \delta_x - \pi \rVert_{TV} < \varepsilon \quad \forall x \in \X \Big\}
    \]
\end{block}

Since
\begin{equation*}\tag{Pinsker inequality}
\lVert \nu - \pi \rVert^2_{TV} \le \frac{1}{2} \mathcal{H}(\nu)
\end{equation*}
we can use MLSI to bound $\tau_{mix}$.

\begin{block}{Convergence to equilibrium}
    Assume $\Ric(\X, Q, \pi)\ge 0$ and the diameter of $(\X,d_{\W})$ is $\le D$. If a MLSI with constant $\lambda= \frac{c}{D^2}$ holds, then
    \[
    \tau_{mix}(\varepsilon) \le D^2(1/8-c\log\varepsilon)    \]
    for some universal constant $c$.
\end{block}
\end{frame}

\begin{frame}{Modified Log-Sobolev inequality for the zero-range process}
    \begin{block}{Diameter bound}
        There is a constant $c>0$ such that the diameter of $(\X_{K,L}, d_\W)$ is bounded by $cK\sqrt{L\log L}$
    \end{block}

    So, for the zero range process with constant uniform rates, a MLSI with constant $\frac{c}{K^2 L \log L}$ holds. 

    \begin{block}{Convergence of the ZRP}
        The total variation mixing time of the zero range process on $\X_{K,L}$ with constant uniform rates is 
        \[
        \tau_{mix}(\varepsilon) \le K L \log L \bigg(\frac{1}{8} - \frac{\log \varepsilon}{c}\bigg)
        \]
        for some universal constant $c$.
    \end{block}
\end{frame}  

\section{Further inequalities}
\tableofcontents[currentsection]
\subsection{Poincaré inequality and the spectral gap}
\begin{frame}{Poincaré inequality}
    % some facts on spectral gap (see notes on log sobolev)
\end{frame}
\subsection{Isoperimetric inequality and Cheeger constant}
\begin{frame}{Isoperimetric inequality and Cheeger constant}
    % notes from prof
\end{frame}

\section{Other applications}
\begin{frame}{example 1}
    % notes from prof
\end{frame}
\begin{frame}{example 2}
    
\end{frame}

% other examples of what this theoreitical framework allows us to get    

\end{document}
