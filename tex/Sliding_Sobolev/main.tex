\documentclass{beamer}

% Set up
\usepackage{xcolor}
\usepackage{graphicx}
\usepackage{amsmath}
\usepackage{graphicx}
\usepackage{amsthm}
\usepackage{mathrsfs}  
\usepackage{amsfonts}
\usepackage{amssymb}
\usepackage{amsmath}
% \usepackage[utf8]{inputenc}

% Beamer setup
\usetheme{default}
\usecolortheme{default}
% \AtBeginSection[]
% {
%   \begin{frame}
%     \frametitle{Table of Contents}
%     \tableofcontents[currentsection]
%   \end{frame}
% }
\usefonttheme[onlymath]{serif}

% Math commands
\newcommand{\1}{\mathbbm{1}}
% measures / sets of measures:
\newcommand{\mH}{\mathscr{H}}
\newcommand{\mL}{\mathscr{L}}
\newcommand{\mP}{\mathscr{P}}
\newcommand{\cH}{\mathcal{H}} % entropy
\newcommand{\cT}{\mathcal{T}} % topology
\newcommand{\cB}{\mathcal{B}} % borel
\newcommand{\W}{\mathcal{W}} % new metric W
\newcommand{\X}{\mathcal{X}} % space X
\newcommand{\nabladot}{\nabla\cdot} % discrete divergence operator
\newcommand{\A}{\mathcal{A}} % action
\newcommand*\de{\mathop{}\!\mathrm{d}}
\DeclareMathOperator*{\argmin}{arg\,min}
\DeclareMathOperator*{\argmax}{arg\,max}
\DeclareMathOperator*{\supp}{supp}
\DeclareMathOperator*{\id}{id}
\DeclareMathOperator{\imm}{Im}
\DeclareMathOperator{\lalpha}{\overline{\alpha}}
\DeclareMathOperator{\aconv}{\alpha \textit{x}+\overline{\alpha}\textit{y}}
\DeclareMathOperator{\bconv}{\beta \textit{x}+\overline{\beta}\textit{y}}

%%%%%%%%%%%%%%%%%%%% from the paper
\newcommand{\R}{\ensuremath{\mathbb{R}}}
\newcommand{\Z}{\ensuremath{\mathbb{Z}}}
\DeclareMathOperator{\Ric}{Ric}
\DeclareMathOperator{\vol}{vol}

\def\diam{\mathop{\mathrm{diam}}\nolimits}
\def\vol{\mathop{\mathrm{vol}}\nolimits}
\def\area{\mathop{\mathrm{area}}\nolimits}
\def\Hess{\mathop{\mathrm{Hess}}\nolimits}
\def\supp{\mathop{\mathrm{supp}}\nolimits}
\def\Ric{\mathop{\mathrm{Ric}}\nolimits}
\def\Id{\mathop{\mathrm{Id}}\nolimits}
\def\tr{\mathop{\mathrm{tr}}\nolimits}
\def\ac{\mathop{\mathrm{ac}}\nolimits}
\def\div{\mathop{\mathrm{div}}\nolimits}
%%%%%%%%%%%%%%%%%%%%%%%%%%%%%%%%%%%%



%Information to be included in the title page:
\title[Inequalities for Markov chains with non-neg Ricci]{Poincaré, modified logarithmic Sobolev and
isoperimetric inequalities for Markov chains with
non-negative Ricci curvature \\~\\ \small M. Erbar and M. Fathi} 
\author[Matilde Dolfato]{\small Matilde Dolfato \\ Mentor: Prof. Elia Brué}
\institute[]{Università Bocconi  \par Visiting Student Initiative - BIDSA}
\date{\today}

\begin{document}

\frame{\titlepage}
% % comment on title

\begin{frame}{Overview of the talk}
    \tableofcontents
\end{frame}

\section{Main result: *high-level sentence* for the zero-range process}
\subsection{The zero-range process}
\begin{frame}{The zero-range process}
    
\end{frame}
\subsection{Interpretation of the result *or other high-level sentence*}
\begin{frame}{Main result and interpretation}
    % HIGH LEVEL dont even tell them it's called log Sobolev or just mention
\end{frame}
\section{From geometry to inequalities}
\subsection{Setup}
\begin{frame}{Setup}
    
\end{frame}
\begin{frame}{Distance $\W$}
    
\end{frame}
\subsection{Ricci for discrete processes}
\begin{frame}{Defining Ricci}
    
\end{frame}
\begin{frame}{Ricci for the zero-range process}
    
\end{frame}
\subsection{Modified Logarithmic Sobolev inequality for discrete processes with non-neg Ricci}
\begin{frame}{The Modified Log-Sobolev inequality}
    
\end{frame}
\begin{frame}{Modified Log-Sobolev inequality for the zero-range process}
    
\end{frame}

\section{Further inequalities}
\subsection{Poincaré inequality and the spectral gap}
\begin{frame}{Poincaré inequality}
    
\end{frame}
\subsection{Isoperimetric inequality and Cheeger constant}
\begin{frame}{Isoperimetric inequality}
    
\end{frame}

\section{Other applications}
\begin{frame}{example 1}
    
\end{frame}
\begin{frame}{example 2}
    
\end{frame}

% other examples of what this theoreitical framework allows us to get    

\end{document}
