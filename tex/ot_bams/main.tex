\documentclass{beamer}

% Set up
\usepackage{xcolor}
\usepackage{graphicx}
\usepackage{amsmath}
\usepackage{graphicx}
\usepackage{amsthm}
\usepackage{mathrsfs}  
\usepackage{amsfonts}
\usepackage{amssymb}
\usepackage{amsmath}
\usepackage{hyperref}
% \usepackage[utf8]{inputenc}

% Beamer setup
\usetheme{default}
\usecolortheme{default}
% \AtBeginSection[]
% {
%   \begin{frame}
%     \frametitle{Table of Contents}
%     \tableofcontents[currentsection]
%   \end{frame}
% }
\usefonttheme[onlymath]{serif}

% Math commands
\newcommand{\1}{\mathbbm{1}}
% measures / sets of measures:
\newcommand{\mH}{\mathscr{H}}
\newcommand{\mL}{\mathscr{L}}
\newcommand{\mP}{\mathscr{P}}
\newcommand{\cH}{\mathcal{H}} % entropy
\newcommand{\cT}{\mathcal{T}} % topology
\newcommand{\cB}{\mathcal{B}} % borel
\newcommand{\W}{\mathcal{W}} % new metric W
\newcommand{\X}{\mathcal{X}} % space X
\newcommand{\nabladot}{\nabla\cdot} % discrete divergence operator
\newcommand{\A}{\mathcal{A}} % action
\newcommand*\de{\mathop{}\!\mathrm{d}}
\DeclareMathOperator*{\argmin}{arg\,min}
\DeclareMathOperator*{\argmax}{arg\,max}
\DeclareMathOperator*{\supp}{supp}
\DeclareMathOperator*{\id}{id}
\DeclareMathOperator{\imm}{Im}
\DeclareMathOperator{\lalpha}{\overline{\alpha}}
\DeclareMathOperator{\aconv}{\alpha \textit{x}+\overline{\alpha}\textit{y}}
\DeclareMathOperator{\bconv}{\beta \textit{x}+\overline{\beta}\textit{y}}

%%%%%%%%%%%%%%%%%%%% from the paper
\newcommand{\R}{\ensuremath{\mathbb{R}}}
\newcommand{\Z}{\ensuremath{\mathbb{Z}}}
\DeclareMathOperator{\Ric}{Ric}
\DeclareMathOperator{\vol}{vol}

\def\diam{\mathop{\mathrm{diam}}\nolimits}
\def\vol{\mathop{\mathrm{vol}}\nolimits}
\def\area{\mathop{\mathrm{area}}\nolimits}
\def\Hess{\mathop{\mathrm{Hess}}\nolimits}
\def\supp{\mathop{\mathrm{supp}}\nolimits}
\def\Ric{\mathop{\mathrm{Ric}}\nolimits}
\def\Id{\mathop{\mathrm{Id}}\nolimits}
\def\tr{\mathop{\mathrm{tr}}\nolimits}
\def\ac{\mathop{\mathrm{ac}}\nolimits}
\def\div{\mathop{\mathrm{div}}\nolimits}
%%%%%%%%%%%%%%%%%%%%%%%%%%%%%%%%%%%%



%Information to be included in the title page:
\title{Relaxed intro to Optimal Transport} 
\author[BAMS]{BAMS}
\date{\today}

\begin{document}

\frame{\titlepage}
    
\begin{frame}{Introduction}
    \begin{itemize}
        \item Problem of transporting mass in an efficient way
        \item Monge 1781: most efficient way of transporting soil from the ground to a given place
        \item Kantorovich 1975: Nobel prize for Economics ``for their contributions to the theory of optimum allocation of resources"
        \item Many formulations: we will focus on two, discrete and continuous versions
        \item Many applications!!!
    \end{itemize}

\begin{columns}[t]
\begin{column}{0.6\textwidth}
\centering
  \includegraphics[scale=.3]{deblais_remblais.png}
\end{column}
\begin{column}{0.2\textwidth}
\centering
  \includegraphics[scale=.25]{gaspard.png}
\end{column}%
\begin{column}{0.3\textwidth}
\centering
  \includegraphics[scale=.25]{leonard.png}
\end{column}
\end{columns}
\pause
\centering
Disclaimer: attempted to be a soft and not 100\% formal introduction to OT! :) 
\end{frame}


\begin{frame}[allowframebreaks]{Why are we doing this}
    \begin{figure}
        \centering
        \includegraphics[width=1\linewidth]{cafe1.png}
        \caption{Supplying all cafes from bakery 1}
        \label{fig: cafe-ineff}
    \end{figure}

        \begin{figure}
        \centering
        \includegraphics[width=1\linewidth]{cafe2.png}
        \caption{Organizing supply using optimal transport!}
        \label{fig: cafe-eff}
    \end{figure}

    \begin{figure}
        \centering
        \includegraphics[width=1\linewidth]{image.png}
        \caption{Optimally allocating color}
        \label{fig: color}
    \end{figure}
\end{frame}

\begin{frame}[allowframebreaks]{Convexity}
    \begin{block}{Convex Set}
        A set $C \subseteq \mathbb{R}^n$ is convex if for any $x,y \in C$ and any $\lambda \in [0,1]$, we have
        \[
        \lambda x + (1-\lambda)y \in C.
        \]
        Intuitively, any line segment between two points in $C$ stays inside $C$.
    \end{block}

    \begin{block}{Extremal Points}
        A point $x \in C$ is an \emph{extremal point} of a convex set $C$ if
        \[
        x = \lambda y + (1-\lambda)z \,\, \Rightarrow\,\, y = z
        \]
        for $y,z \in C$ and $\lambda \in (0,1)$. 

        E.g. the extremal points of a polygon are its vertices.
    \end{block}

    \begin{block}{Convex Function}
        A function $f:C\to\mathbb{R}$ defined on a convex set $C$ is convex if
        \[
        f(\lambda x + (1-\lambda)y) \le \lambda f(x) + (1-\lambda)f(y)
        \]
        for all $x,y \in C$, $\lambda \in [0,1]$.  
        Geometrically, its graph lies below the chord joining any two points.
    \end{block}

    \begin{block}{Polyhedra}
        A polyhedron is the intersection of finitely many hyperplanes, i.e. for $b\in\R^n$ and $A\in\R^{n\times n}$
        \[
        P = \{x \in \mathbb{R}^n : Ax \le b\}.
        \]
        If $P$ is bounded, it is a polytope.
    \end{block}

    
    \begin{block}{Polytope}
        A polytope is a bounded convex set obtained as the convex hull of finitely many points.  
        Example: a triangle, cube, or any convex polygon/polyhedron.
    \end{block}

    \begin{block}{Convex Optimization}
        A convex optimization problem is one of the form
        \[
        \min_{x \in C} f(x)
        \]
        where $C$ is convex and $f$ is convex.  
        Nice property: any local minimum is also a global minimum.
    \end{block}
\end{frame}

\begin{frame}{Discrete Monge formulation}
\begin{itemize}
    \item Two clouds of points $\{x_i\}_{i=1,\cdots,n}$, $\{y_i\}_{i=1,\cdots,n}$ 
    \item Two probability measures (distribution of masses) on them
    \item The cost of transporting mass $x_i$ to $y_j$ is $C_{ij}$ 
\end{itemize}
\begin{block}{Permutation}
    Given an interval $I\subset\mathbb
    N, I =\{1,\cdots,n\}$, a permutation is a bijective function $\sigma:I \to I$. \\
    We indicate $\sym(I)$ or just $\sym(n)$ the set of all permutations on $I=\{1,\cdots,n\}$.
\end{block}
\pause
\begin{block}{Monge's formulation}
    \begin{equation}\tag{D1}\label{D1}
        \min_{\sigma \in \sym(n)} \sum_{i,j} C_{i, \sigma(j)}
    \end{equation}
\end{block}

\end{frame}

\begin{frame}{Discrete Kantorovich formulation}

Kantorovich's extension: we are allowed to ``divide" masses! yay!
\vspace{20pt}

\begin{block}{Bistochastic matrices}
    A matrix $P\subset\R_+^{n\times n}$ is a bistochastic matrix if $\sum_jP_{ij}=1$, $\sum_iP_{ij}=1$, i.e. its rows and its columns sum to 1. \\
    We indicate $B_n$ the set of bistochastic matrices in $\R_+^{n\times n}$.
\end{block}
\pause
\begin{block}{Kantorovich's formulation}
        \begin{equation}\tag{D2}\label{D2}
            \min_{P\in B_n} \sum_{ij}P_{ij}C_{ij}
        \end{equation}
\end{block}
\end{frame}

\begin{frame}{Connecting Monge and Kantorovich}
    
We can rewrite permutations as in Monge's formulation via permutations matrices \[P_n:= B_n \cap \{0,1\}^{n \times n}\]
Then Monge's problem becomes:
\begin{equation}\tag{D1$^\prime$}
    \min_{P\in P_n} \sum_{ij}P_{ij}C_{ij}
\end{equation}
\pause
\begin{alertblock}{Important remarks}
\begin{enumerate}
    \item $B_n \subset [0,1]^{n \times n}$ is a polyhedra
    \item $P_n= \rm Ext(B_n)$ (Von Neumann)
    \item Kantorovich formulation is a \textbf{convex relaxation} of Monge's one 
    \item OT problem as in \ref{D2} can be solved efficiently via linear programming
\end{enumerate}
\end{alertblock}
\end{frame}

\begin{frame}{Probability measures}

% sigma algebra ?

    \begin{block}{Probability measure}
     A probability measure on (a $\sigma$-algebra $\mathcal{A}$ of) a set $X$ is a measure $\mu:\mathcal{A}\to [0,1]$ such that $\mu(X)=1$. We indicate as $\mathcal{P}(X)$ the set of measures on $X$. 
    \end{block}
\pause
    \begin{block}{Push-forward measure}
        Given a function $T:X \to Y$, we define the push forward operator $T_\#: \mathcal{P}(X) \to \mathcal{P}(Y)$ by
    \[
    T_\#\mu(A) = \mu(T^{-1}(A)) \quad \forall A \in \mathcal{B}(Y)
    \]
    and call $T_\#\mu$ push forward measure.
    \end{block}

    The push forward measure is the measure assigning to a set the measure of its \textbf{pre-image} according to a map $T$.
\end{frame}

\begin{frame}{Transport via functions}
Given
    \begin{itemize}
    \item two spaces $X$ and $Y$ 
    %what spaces are these hihi? 
        \item probability measures $\mu \in \mP(X)$, $\nu \in \mP(Y)$
        \item  cost function $c:X\to Y$
    \end{itemize} 
    \begin{block}{Monge's formulation}
        \begin{equation}\tag{C1}
            \min_{T | T_\#\mu=\nu} \int_Xc(x,T(x)) \de \mu(x)
        \end{equation}
    \end{block}
\end{frame}

\begin{frame}[allowframebreaks]{Transport via transport plans}
\begin{block}{Transport plan}
A transport plan is a probability measure $\pi \in \mP(X \times Y)$ such that $\pi(A\times Y) = \mu(A), \pi(X\times B) = \nu(B)$. \\
We indicate $\Gamma (\mu, \nu)$ the set of all transport plans for $\mu$ and $\nu$.
\end{block}

Transport plans are another way of carrying mass from $X$ to $Y$, and in particular $\pi(A\times B)$ is the mass that was in $A \subset X$ and has been sent to $B \subset Y$.\\
\vspace{5 pt}
For stats buddies: transport plans are like joint probability measures! 

\begin{block}{Kantorovich's formulation}
\begin{equation}\tag{C2}
    \min_{\pi \in \Gamma(\mu,\nu)} \int_{X\times Y} c(x,y) \,\,\de{\pi}(x,y)
\end{equation}
\end{block}
\end{frame}

\begin{frame}{Connecting Monge and Kantorovich, again}
    For any transport \textit{map} $T$, there exists a transport \textit{plan} $\pi_T$:
    \[
    T \longmapsto (\id \times\, T)_\#\mu =: \pi_T
    \]
    Moreover, it can be checked $\mathcal{C}(T) = \mathcal{C}(\pi_T)$ where $\mathcal{C}$ is the total cost attained by a transport map/plan. \\
\vspace{10 pt}
Thus, 
\[
\text{inf}_{M} \ge \text{inf}_{K}
\]
\end{frame}

\begin{frame}{Useful resources}
\begin{itemize}
\item Brué Ambrosio Semola Lecture on Optimal Transport
\item \href{http://elenaher.dinauz.org/B07D.StFlour.pdf}{Villani Optimal Transport Old and New}
    \item \href{https://optimaltransport.github.io/book/}{Peyré and Cuturi Computational Optimal Transport}
    \item \href{https://optimaltransport.github.io/slides-peyre/course-note-ot-basics.pdf}{Peyré slides on discrete part}
    \item \href{https://matildedol.github.io/files/Ricci.pdf}{Mati's notes chapter 3}
\end{itemize}
Beware: there's a sea out there, only check out topics we went over or ask for more!
\end{frame}

\end{document}