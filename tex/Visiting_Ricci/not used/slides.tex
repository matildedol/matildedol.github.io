prima cosa che faccio:
 1)
RICCI CURVATURE

OPTIMAL TRANSPORT 

BEHAVIOR OF MEASURES ON MANIFOLD 


key facts:
- optimal transport inherits the (geo)metric structure of the underlying space (and allows to define a distance between measures and how to transport one into the other). in particular, the ricci curvature is crucial in explaining optimal transport on the manifold. 

- optimal transport is useful for us because: \textit{Ric tells us something about optimal transport of markov chains. what is it and why is it inetresting?}

2)
capire quanto andare nel deep di questa teoria, prerequisiti

3) 
Notion of Ricci > notion of Ricci for metric spaces > notion of Ricci for metric measure spaces, cioè iff convexity of entropy perché applicable to measure spaces > necessita comunque geodetiche > non ci sono in uno spazio discreto >  nuova nozione per discrete spaces

4) perché vogliamo sta cosa????
-motivation:
\cite{lott2009ricci} we study optimal t of measure to study pdes

5) iniziare con descrizione metrica, ecc 

setting pre metrica: usa pg 3


dopo la definizione di W:
cose da dire:
- è una distanza riemanniana con la struttura sotto definita, cioè con il tangent space definito così (scopriamo che la relazione tra rho e psi è che il gradiente di psi vive nello spazio tangente a rho) 