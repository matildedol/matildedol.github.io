Now, we endow our sample space with a distance $d$ getting a metric space $(\Omega,d)$ and indicate $\mathcal{P}_2(\Omega)$ the set of probability measures on $\Omega$ endowed with the ($L^2$-) Wasserstein distance. 
This notion is at the basis of optimal transport. 

\begin{definition}[Wasserstein distance]
Given a metric space $(\Omega, d)$ and two probability measures $\mu_0, \mu_1 \in \mathcal{P}_2(\Omega)$, the $L^2$-Wasserstein distance $W_2: \mathcal P_2(\Omega)\times\mathcal P_2(\Omega)\to[0,\infty)$ between them is defined as
    \begin{align*}
        W_2 (\mu_0, \mu_1) := \Biggl[\inf_{\pi \in P(\Omega \times \Omega)} \int d^2 (x,y) \, d \pi (x,y)\Biggr]^2
    \end{align*}
\end{definition}

where $\pi$ is a \emph{coupling} of $\mu_0$ and $\mu_1$, i.e. a joint probability measure. 

We can also define a Wasserstein geodesic. 
\begin{definition}[Wasserstein geodesic]
        Given two measures $\mu_0, \mu_1$, the W. geodesic between them is a curve $\gamma(t) $ s.t. 
    \begin{align*}
        &\gamma(0)=\mu_0, \,\, \gamma(1)=\mu_1\\
        &W_2 (\gamma(s), \gamma(t)) = |t-s|\, W_2( \mu_0, \mu_1) \quad \forall s,t \in [0,1]
    \end{align*}
\end{definition}



We also clarify what being convex with respect to the $L^2$-Wasserstein geometry means.
\begin{definition}[Wasserstein convexity]
        Given a manifold $(M,g,\vol_g)$ and a function $f$, and two measures $\mu_0, \mu_1$ and their W. geodesic $\mu(t)$, $f$ is W. convex if
    \begin{align*}
        F(\mu(t))\le(1-t)F(\mu_0 )+ tF(\mu_1)\,\, \forall t \in [0,1]
    \end{align*}
\end{definition}

Now, the basic result in optimal transport on Euclidean spaces is the following. We denote as $\mathcal{P}_c(\Omega)$ the space of probability measures on a compact support, and $\mathcal{P}^{ac}(\Omega, m)$ the space of absolutely continuous probability measures with respect to a measure $m$.

\begin{theorem}[Optimal transport - Euclidean space]
    Given $\mu_o,\mu_1 \in \mathcal{P}_c(\Omega)$ probability measures with compact support, with $\mu_0 \in \mathcal{P}^{ac}(\Omega, \vol_n)$, there is a convex function $f:\R^n \to \R$ such that the map
    \[
    \mathcal{F}_t(x):=(1-t)x+t\nabla f(x), \quad t\in[0,1]
    \]
    gives the unique optimal transport from $\mu_0$ to $\mu_1$. Precisely, $t \mapsto\mu_t:=(\mathcal{F}_t)_{\sharp}\mu_0$ is the unique minimal geodesic from $\mu_0$ to $\mu_1$ with respect to the $L^2$-Wasserstein distance. 
\end{theorem}

Also, $f$ is twice differentiable a.e. by Alexandrov's theorem, hence $\mathcal{F}_t$ makes sense and is differentiable. The theorem above is known as Brenier's theorem. 

Another core result in optimal transport is the Brunn-Minkowski inequality, that essentially states that \textbf{the volume measure to the power of $1/n$ is concave}. 
\begin{corollary}[Brunn-Minkowski inequality]
    For any measurable sets $A,B \subset \R^n$ and $t \in [0,1]$, we have
    \[
    \vol_n\bigl((1-t)A+tB\bigr)^{1/n} \ge (1-t) \vol_n(A)^{1/n} + t \,\vol_n(B)^{1/n}
    \]
\end{corollary}


Now, the problem of optimal transport on a Riemannian manifold is described in the same way as for Euclidan spaces. 
\begin{theorem}[Optimal transport - Riemannian manifold]
    Given $\mu_o,\mu_1 \in \mathcal{P}_c(M)$ probability measures with compact support, with $\mu_0=\varrho_0 \vol_g \in \mathcal{P}^{ac}(M,\vol_n)$, there exists a ($d^2/2$)-convex function $f:M\to\R$ such that
\[
\mu_t:=(\mathcal{F}_t)_\sharp \mu_0 \]\[
\text{with } \mathcal{F}_t(p) := \exp_p[t\nabla f(p)], \,\, t\in[0,1]
\]
gives the unique minimal geodesic from $\mu_0$ to $\mu_1$, w.r.t. the Wasserstein distance defined for the distance $d_g$, induced by the metric $g$. 
\end{theorem}

We also recall the definition of (Shannon) entropy with respect to the measure $\vol_g$.
\begin{definition}[Boltzmann-Shannon entropy]
Given an absolutely continuous measure $\mu = \varrho \,\vol_g \in \mathcal{P}^{ac}(M,\vol_g)$,
\[
    \cH_{\vol_g}(\mu) := \int_M \varrho \log \varrho \,\, d\vol_g \quad \in \R
    \]
\cite{Ohta_2014}
\end{definition}

\begin{remark}
    When we consider $\mu = \varrho \vol_g$, we are saying that a probability measure can be seen as a density $\varrho$ times a measure (volume). We are interested in integrating the density of the probability measure, against the volume. 
\end{remark}
