\section{Riemannian geometry}

Seminar 15 apr, Prof. Pigati. [Next meeting is May 13, then May 27 ]
% THERE IS NO AUDIO :((((
% connections 
% concepts in riem. geometry 

Connections:
a way to perform derivatives on a manifold

\begin{definition}[Section of a map] A smooth section $s:M \to E$ is a smooth map such that $s(p) \in \pi^{-1}(p) \,\, \forall p\in M$
\end{definition}
When $E = TM$, a section is called a vector field. 
(Recall $E$ is our vector bundle with associated map $\pi$, and $X=\sum_{i=1}^n X^i\frac{\partial}{\partial x_i}$.

\begin{definition}[Connection]
    A connection on $E$, denoted $\nabla$, takes a section $s$ of $E$ and a vector field $X$, and gives a new section $\nabla_Xs$ that satisfies the following properties:
\begin{itemize}
    \item $C^\infty(M)$-linear in $X$: $\nabla_{fX+gY} s=f\nabla_X s + g\nabla_Y s \quad \forall f,g: M \to \mathbb{R}$
    \item $\mathbb{R}$-linear in s
    \item Leibniz rule: $\nabla_X(fs) = f \nabla_x s + X(f) s$
\end{itemize}
Where $X(f)$ indicates differentiating $f$ along direction $X$.  
\end{definition}

\begin{remark}
    $\nabla_X s(p)$ depends only on $s|_U$ for an arbitrary small neighborhood $U$ and on $X(p)$.
\end{remark}

\begin{remark}
    In a trivialization $\pi^{-1} \cong \mathbb{R}^k \times U$ we always have the following structure:
    \[
    s(x) = (v(x), x)
    \]
    \[
    \nabla_X s(x) = Ds(x) [X(x)] + X^iA_i(x) s(x)
    \]
    Where $A$ is a $k\times k$ matrix, not necessarily invertible. 
\end{remark}

\begin{remark}
    If $\nabla$ is a connection, any other connection has the form $\tilde{\nabla} = \nabla + \alpha$ where $\alpha(x):\pi^{-1}(x) \otimes T_x^*M \to \pi^{-1}(x)$ is a linear map. Namely, $\tilde \nabla_X s = \nabla_X s + \alpha(X) s$, where $\alpha(X) s$ is the ``0-th order term". 
\end{remark}

Hence, the space of all connections is an \textbf{affine space}. 

In computations, assuming $U$ trivializes $E$ and is included in a chart, we can write \[\nabla_{\frac{\partial}{\partial x_i}}e_j= \sum_{l=1}^{\text{rank}} \Gamma_{ij}^l e_l\] where $\{e_1, \ldots, e_k\}$ is a canonical basis of $\mathbb{R}^k \cong \pi^{-1}(x)$. $\Gamma_{ij}^l (x)$ are called \emph{Christoffel symbols}. 

$A_i(x)$ has coefficient $\Gamma_{ij}^l $ in position $(l,j)$.

\[
\nabla_{\frac{\partial}{\partial x^i}}
\begin{pmatrix}
c_1(x)\\
\vdots\\
c_k(x)
\end{pmatrix}
=
\begin{pmatrix}
\frac{\partial c_1}{\partial x^i}\\[6pt]
\vdots\\[6pt]
\frac{\partial c_k}{\partial x^i}
\end{pmatrix}
\;+\;
A_i
\begin{pmatrix}
c_1(x)\\
\vdots\\
c_k(x)
\end{pmatrix}\,.
\]

\begin{remark}
    For $E=TM$ we just need to take $U$=domain of chart. Then $TM|_U$ is trivialized by $c_1(x)\frac{\partial}{\partial x^1} + \ldots + c_n(x)\frac{\partial}{\partial x^n}$ and $\nabla_{\frac{\partial}{\partial x^i}} (\frac{\partial}{\partial x^j}) = \sum_{l=1}^n \Gamma_{ij}^l \frac{\partial}{\partial x^l}$
\end{remark} 

\subsection{Parallel Transport}
Given a smooth curve $\gamma: [0, T] \to M$, a section along $\gamma$ is $s:[0,T] \to E$ (smooth) s.t. $s(t) \in \pi^{-1}(\gamma(t))$

\begin{example}
    Given any $v_0 \in \pi^{-1}(\gamma(0)$ there is a unique section $s(t)$ along $\gamma$ s.t. $\nabla_{\gamma'(t)} s(t) = 0.$ What does $\nabla_{\gamma'(t)}$ mean? In Rimeannian geometry, it is called the \emph{covariant derivative} along a curve $\gamma$.
\end{example}

Locally in $t$ we can always write
\[
s(t) = \sum_m c_m(t) s_m(\gamma(t))
\]
with $s_m$ section in the usual sense, define near $\gamma(t)$, and we let \[\nabla_{\gamma'(t)}:= \sum c_m(t) \nabla_{\gamma'(t)} s_m + \sum c_m'(t) s_m(\gamma(t))\]

What we get $v_0(s(0) \mapsto s(T))$ is called \emph{parallel transport} along $\gamma$ and gives an isomorphism $\pi^{-1}(\gamma(0)) \overset{\sim}\longrightarrow \pi^{-1}(\gamma(T)) $. However, two curves will produce different isomorphisms $\pi^{-1}(p) \to \pi^{-1}(q)$. 

What is the \emph{curvature} of a connection? It measure the failure of parallel transport along loops to be an identity. 
\[F_\nabla \Bigl(\frac{\partial}{\partial x^i}, \frac{\partial}{\partial x^j} \Bigr) s := \nabla_{\frac{\partial}{\partial x^i}} (\nabla_{\frac{\partial}{\partial x^j}} s ) - \nabla_{\frac{\partial}{\partial x^j}} (\nabla_{\frac{\partial}{\partial x^i}} s) = B_{ij} s\]
is a 0-order thing. 
Hence, parallel transport = $\text{Id} + \varepsilon^2 \cdot F_\nabla \bigl(\frac{\partial}{\partial x^i}, \frac{\partial}{\partial x^j}\bigr)$

\subsection{Riemannian manifold}
A Riemannian manifold $(M,g)$ is a manifold endowed with a positive definite scalar product $g(p)$ on each $T_pM$, depending smoothly on $p$. 

\begin{example}
    IF $M \subseteq \mathbb{R}^N$, we can take $g(p):=$ restriction of the inner product in $\mathbb{R}^N$, e.g. $S^N \subset \mathbb{R}^{N+1}$
\end{example}

\begin{theorem}
    On $TM$, $ $ connection, called \emph{Levi-Civita connection}, s.t.
    \begin{itemize}
        \item$ X(g(Y,Z)) = g(\nabla_XY,Z) + g(Y, \nabla_X Z)$
    \end{itemize}
\end{theorem}

The Riemannian tensor on ($M, g$) is defined as
\[
R(X,Y)Z := \nabla_X\nabla_YZ - \nabla_Y \nabla_X Z
\]
and is a section of $TM \otimes T^*M \otimes T^*M \otimes T^*M$.

[...]