[Missing lecture May 13]
% va rivista con registrazione
\\

Recall that our framework is that of $(M, g, \nabla)$
\\

[two theorems explained last time]
[Fenkel duality explained last time]

\begin{definition}[Bregman divergence]
    Consider $p, q \in M, \, \theta p = \theta(p), \,\theta q=\theta(q)$, and a strongly-convex function $f$. Then we define Bregman divergence as
    \[
    D_f(p,q):= f(\theta p) - f(\theta q) - \nabla f(\theta p)(\theta p-\theta q)
    \]
\end{definition}

\begin{remark}
    \begin{itemize}
        \item[(i)] $D_f(p,q) \ge 0 \,\, \forall p,q \in M$ and $D_g = 0$ if and only if $p=q$;
        \item[(ii)] $D_f$ is not symmetric.
    \end{itemize}
\end{remark}

Analogously, we can define the \emph{dual divergence} as \[
D_{f^*}(p,q):= f^*(\theta^* p) - f^*(\theta^* q) - \nabla^* f^*(\theta^* p)(\theta^*p-\theta^* q)
\]

\begin{proposition}
   $ D_f(p,q) = D_{f^*}(q,p) \quad \forall p,q \in M$

\begin{proof}
Notice that $\nabla f(\theta q) = \theta^*q$ by definition of $\theta^*$ (Definition \ref{}).
\begin{align*}
         D_f(p,q)&=f(\theta p) - f(\theta q) - \nabla f(\theta p)(\theta p-\theta q)  \\
         &= f(\theta p) - f(\theta q) + \theta q \theta^* q- \theta^*q\theta p\\
         &= f(\theta p) + f^*(\theta^* q) - \theta^* q\theta p
\end{align*}
Where the last equality comes from $-f(\theta q) + \theta q \theta^* q= f^*(\theta^* q)$ by Definition \ref{def: fenkel}. By playing the same game with $D_{f^*}(p,q)$ we get
\[
D_{f^*}(p,q) = f^*(\theta^* p) + f(\theta q) - \theta q\theta^* p
\]
by using the third point in Definition \ref{def: fenkel} and that the * operation is an involution. The result follows. 
\end{proof}
\end{proposition}

\subsubsection{Generalized Pythagorean Theorem}
Consider3 point $p,q,r \in M$ and 3 curves connecting them. We look at the remainder of the divergence between the 3 in a cycle, in the same fashion of what the Pythagorean theorem does with Euclidean distances. Indeed, we can think of the divergence as a generalized \emph{squared} distance.

\begin{align*}
D_f(p,q)+D_f(q,r)-D_f(p,r)
= &
\bigl[f(\theta_p)-f(\theta_q)-\theta_q^*(\theta_p-\theta_q)\bigr] + 
\bigl[f(\theta_q)-f(\theta_r)-\theta_r^*(\theta_q-\theta_r)\bigr] \\
&-\bigl[f(\theta_p)-f(\theta_r)-\theta_r^*(\theta_p-\theta_r)\bigr]\\
= & \,\theta_r^*(\theta_p-\theta_r)-\theta_q^*(\theta_p-\theta_q)\\
=&\,(\theta_r^*-\theta_q^*)(\theta_p-\theta_q)
\end{align*}

Hence, the remainder of this commutator is the inner product between $(\theta_r^*-\theta_q^*)  \text{ and }(\theta_p-\theta_q)$. Let us make sense of this number. 

Consider the $\nabla$-geodesic connecting $p$ and $q$:
\[
\gamma_{p,q}(t)= \theta p(\-t) + t\theta q
\]
and the $\nabla^*$-geodesic connecting $q$ and $r$:
\[
\gamma_{q,r}(t)= \theta^* q(1-t) + t\theta^* r
\]
Now, velocities of the two curves are explicit:
\[
\dot \gamma _{p,q}(t) = \sum_i (\theta q - \theta p)_i \frac{\partial}{\partial \theta_i}
\]
\[
\dot\gamma _{q,r}= \sum_i (\theta^*r - \theta^* q)_i\frac{\partial}{\partial \theta^*_i}
\]
and similarly for $\dot \gamma_{q,r}$.

By inspecting Figure \ref{fig: pyth}, we see that the pink angle between the two geodesics is the inner product between (minus) the velocity of $\gamma _{p,q}$ at $t=1$ (pink arrow) and the velocity of $\gamma _{q,r}$ at $t=0$ (blue arrow), i.e. 
\[
g\Bigl(-\sum_i (\theta q - \theta p)_i \frac{\partial}{\partial \theta_i}, \,\sum_j (\theta^*r - \theta^* q)_j\frac{\partial}{\partial \theta^*_j}\Bigr) = (\theta_p-\theta_q)(\theta_r^*-\theta_q^*)
\]
Where the result is just the Euclidean inner product between the two quantities. Hence, if $\dot \gamma_{p,q} \perp \dot \gamma_{q,r}$, their inner product equals 0. \\
The result is that if the $\nabla$-geodesic $\gamma_{p,q}$ is orthogonal to the $\nabla^*$-geodesic $\gamma_{q,r}$, then:
\[
D_f(p,q)+D_f(q,r)-D_f(p,r) = (\theta_p-\theta_q)(\theta_r^*-\theta_q^*) = 0\, 
\]
Implying:
\begin{equation}\label{eq: pyth}
    D_f(p,q)+D_f(q,r)=D_f(p,r)
\end{equation}


\begin{remark}
    We can state a more general statement. Indeed, to get result \ref{eq: pyth} we just need that the velocities of the curves connecting $p$ and $r$ to $q$ are orthogonal at $q$ (at $t=1$ and $t=0$ respectively). This means that any two curves connecting $p$ to $q$ and $r$ to $q$, that have the same first-order behavior of the $\nabla$ and $\nabla^*$ geodesics at $t=1$ and $t=0$ respectively, are suitable. 
\end{remark}

This result is widely used in optimization, to characterize minimizers in terms of velocity of geodesics, for instance minimizing the entropy as a specific kind of divergence. 
\begin{figure}
    \centering
    \includegraphics[width=0.3\linewidth]{images/pyth.jpg}
    \caption{}
    \label{fig: pyth}
\end{figure}

[parte di ottimizzazione]
\\

[parte degli esponenziali]