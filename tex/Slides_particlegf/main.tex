\documentclass{beamer}

% Set up
\usepackage{xcolor}
\usepackage{graphicx}
\usepackage{amsmath}
\usepackage{graphicx}
\usepackage{amsthm}
\usepackage{mathrsfs}  
\usepackage{amsfonts}
\usepackage{amssymb}
\usepackage{amsmath}
\usepackage{mathtools}
% \usepackage[utf8]{inputenc}

% Beamer setup
\usetheme{default}
\usecolortheme{default}
% \AtBeginSection[]
% {
%   \begin{frame}
%     \frametitle{Table of Contents}
%     \tableofcontents[currentsection]
%   \end{frame}
% }
\usefonttheme[onlymath]{serif}

% Math commands
\newcommand{\1}{\mathbbm{1}}
% measures / sets of measures:
\newcommand{\mH}{\mathscr{H}}
\newcommand{\mL}{\mathscr{L}}
\newcommand{\mP}{\mathscr{P}}
\newcommand{\cH}{\mathcal{H}} % entropy
\newcommand{\cT}{\mathcal{T}} % topology
\newcommand{\cB}{\mathcal{B}} % borel
\newcommand{\W}{\mathcal{W}} % new metric W
\newcommand{\X}{\mathcal{X}} % space X
\newcommand{\nabladot}{\nabla\cdot} % discrete divergence operator
\newcommand{\A}{\mathcal{A}} % action
\newcommand*\de{\mathop{}\!\mathrm{d}}
\DeclareMathOperator*{\argmin}{arg\,min}
\DeclareMathOperator*{\argmax}{arg\,max}
\DeclareMathOperator*{\supp}{supp}
\DeclareMathOperator*{\id}{id}
\DeclareMathOperator{\imm}{Im}
\DeclareMathOperator{\lalpha}{\overline{\alpha}}
\DeclareMathOperator{\aconv}{\alpha \textit{x}+\overline{\alpha}\textit{y}}
\DeclareMathOperator{\bconv}{\beta \textit{x}+\overline{\beta}\textit{y}}

%%%%%%%%%%%%%%%%%%%% from the paper
\newcommand{\R}{\ensuremath{\mathbb{R}}}
\newcommand{\Z}{\ensuremath{\mathbb{Z}}}
\DeclareMathOperator{\Ric}{Ric}
\DeclareMathOperator{\vol}{vol}

\def\diam{\mathop{\mathrm{diam}}\nolimits}
\def\vol{\mathop{\mathrm{vol}}\nolimits}
\def\area{\mathop{\mathrm{area}}\nolimits}
\def\Hess{\mathop{\mathrm{Hess}}\nolimits}
\def\supp{\mathop{\mathrm{supp}}\nolimits}
\def\Ric{\mathop{\mathrm{Ric}}\nolimits}
\def\Id{\mathop{\mathrm{Id}}\nolimits}
\def\tr{\mathop{\mathrm{tr}}\nolimits}
\def\ac{\mathop{\mathrm{ac}}\nolimits}
\def\div{\mathop{\mathrm{div}}\nolimits}
%%%%%%%%%%%%%%%%%%%%%%%%%%%%%%%%%%%%


\title{On the global convergence of gradient descent using optimal transport\\ 
{\small L. Chizat, F. Bach, NeurIPS 2018}}
\author{Matilde Dolfato \\
{\it Introduction to Real Analysis II}, Prof. G. Savaré}
\date{\today}

\begin{document}
\maketitle

\begin{frame}{Overview}
    \tableofcontents
\end{frame}

\section{Introduction}

\begin{frame}{Introduction}

Classical task in machine learning:
{\color{blue!60!black}
\begin{equation}
     \min_{\alt<2>{\color{blue} \mu \in \mP(\Theta)}{\mu \in \mP(\Theta)}} j(\mu) \qquad j(\mu)= \ell\bigg(\int\phi\de\mu \bigg) + g(\mu)
\end{equation}}

$\ell: \mathcal{F}\to\R_+$ smooth, convex \textit{loss function}

data lives in a large parametrized set with parameters $p\in\Theta \subset \R^d$

$g:\mP(\Theta)\to\R$ optional convex \textit{regularizer}
\\~\\

we look for a linear combination $\phi$ mapping features to labels $\phi(p): x\mapsto \sigma(\sum_i p_i x_i + p_d), \, p\in\Theta$


\pause 
\bigskip
\centering
{\color{blue}  $\star$ Minimize over the space of measures $\mP(\Theta)$ $\star$ }

% however, è intrattabile
% tu lo devi discrettizare (particle gradient descent) e il problema discretizzato non è convesso 


\end{frame}
\begin{frame}{Introduction}
    {\small \begin{equation}\tag{1}
    \min_{\mu\in\mP(\Theta)}j(\mu)= \ell\bigg(\int\phi\de\mu \bigg) + g(\mu) 
    %\quad\longrightarrow \quad \min_{w, p} j_m(w,p) = \ell\bigg(\sum_{i=1}^m w_i \phi(p_i) \bigg) 
\end{equation}} 
    Convex problem, but intractable
\pause

$\Rightarrow$ \alt<2>{\color{blue} discretize}{ discretize} {\color{black} the measure into $m$ particles parametrized by {\it weights} and {\it positions}
\[
\mu_m := \frac{1}{m}\sum_{i=1}^m w_i \delta_{p_i}
\]
{\begin{equation}\label{eq: discrete problem}
\min_{\alt<2>{\color{blue} \substack{w\in \R \\ p \in \Theta^m}}{\substack{w\in \R^m \\ p \in \Theta^m}}} j_m(w,p) \qquad j_m(w,p):= j(\mu_m) 
\end{equation}}}
\pause
Non-convex problem

\centering
{\color{blue}$\star$ Idea: use tools from optimal transport to exploit convexity of (1) to study global convergence in (2) $\star$}
\end{frame}

\begin{frame}{Conceptual key points}
% metti problmea infinto -> problema 
% riformulare il problema in modo da poter applicare tecniche di gradient flow (prima cosa che faremo) 

% poi guardi a un continuous time gradient descent come un gradient flow 

% noti che la formulazione infinito dimensionale è una geenralizzazione del caso discreto (Wasserstein gradient flow generalizes particle gradient flow), quindi sfrutti le belle proprietà di (1) 
\begin{enumerate}
    \item Problem reformulation (lifting) {\color{black}$j \rightarrow f$}
    \item To be able to study gradient flow of {\color{black}$f_m$}, {\color{black}$(\mu_{m,t})_t$}
    \item Look at it as a particular case of infinite-dimensional case 
    \[\color{black} \text{as } m\to\infty, \,(\mu_{m,t})_t \to (\mu_t)_t\]
    \item Exploit convex structure of inf-dim case
    \[
    \lim_{m,t\to\infty} f(\mu_{m,t}) = \min_{\mu_t\in\mP(\Omega)} f(\mu_t)
    \]
    % \item Characterize its limit as a Wasserstein gradient flow of $f$
    % \item Use global convergence of W. gradient flow to study convergence of gradient descent
\end{enumerate}
\end{frame}


\section{Problem re-formulation }
\begin{frame}{Overview}
    \tableofcontents[currentsection]
\end{frame}
\begin{frame}{Lifting}
{\small    \begin{equation} \tag{1}\min_{\nu\in\mP(\Theta)}j(\nu)= \ell\bigg(\int\phi\de\nu \bigg) + g(\nu)\end{equation} }
\vspace{-1cm}
 \begin{center}{ \color{blue!70!black}
$\downarrow\ {\small h_1^{-1}}$}\end{center}
{ \color{blue!70!black}\begin{equation}
\begin{gathered}
\min_{\mu\in\mP(\Omega)} f(\mu) = \ell\bigg(\int\varphi\de\mu\bigg)+\int v\de\mu
\end{gathered}
\end{equation}}

Recovering the lifting:
\begin{itemize}
\item $\Omega=\Theta \times \R$
    \item $\varphi(p,w) = w\phi(p)$
    \item projection map: $h_1:\Theta \times \R \to \Theta, \,h_1(\mu)(p)= \int_\R w \mu(\de w, p) $
    \item $g(\nu) = \inf_{\mu\in h_1^{-1}(\nu)} \int v \de \mu$
    \item then, $\inf_\nu j = \inf_\mu f$
\end{itemize}
\end{frame}

\begin{frame}{New problem}
{\small \begin{equation*}
\min_{\mu\in\mP(\Omega)} f(\mu) = \ell\bigg(\int\varphi\de\mu\bigg)+\int v\de\mu
\end{equation*}}

    Discretized problem:
{\color{blue!70!black}  \begin{equation*}
    \min_{u\in\Omega} f_m(u) := f\bigg(\underbrace{\frac{1}{m} \sum_{i=1} \delta_{u_i}}_{\mu_m}\bigg) = \ell \bigg(\frac{1}{m} \sum_{i=1}^m \varphi(u_i)\bigg) + \frac{1}{m}\sum_{i=1}^m v(u_i)
\end{equation*}}

    {\centering
    {\color{blue} $\star$ Weights $w$ are another coordinate of a particle position $u\in\Omega$ \,\,
$\Rightarrow$ study gradient flow of $f_m$ $\star$}}

\pause
\bigskip
    Assumptions (high-level):
\begin{itemize}
    \item $\ell$ is smooth, with Lispchitz and bounded differential $\de \ell$
    \item $\phi$ is differentiable and $v$ is semiconvex
    \item there is a family of nested closed convex sets $(Q_r)_r$ on which the (sub)derivatives of $\phi,v$ are Lipschitz and grow sublinearly
\end{itemize}

\end{frame}


\section{Apply gradient flow theory}
\begin{frame}{Overview}
    \tableofcontents[currentsection]
\end{frame}
\subsection{Particle gradient flow}

\begin{frame}[allowframebreaks]{Particle gradient flow }
\begin{definition}[Particle gradient flow]
A gradient flow for $f_m$ is an absolutely continuous function $u:\R_+\to\Omega^m$ such that
\[
u'(t) \in -m\,\partial f_m(u(t))
\]

{\color{black}for almost every $t\in\R_+$. }
\end{definition}

Properties:
\begin{enumerate}
    \item[$(i)$] (existence and uniqueness) for any $u(0)\in\Omega^m$ starting point, there exists a unique gradient flow for $f_m$
    \item[$(ii)$] (derivative of $f_m$) for a.e. $t\in\R_+$, it holds 
    \[
    \frac{\de}{\de s} f_m(u(s)) \,\bigg\vert_{s=t} = \,\,- \vert u'(t) \vert^2
    \]
    \item [$(iii)$](form of the velocity) the velocity {\it of the $i$-th particle} $u_i(t)$ is a vector field $v_t:\Omega \to \R^d$ given by $u_i'(t) = v_t(u_i(t))$ where \cite{santambrogio2017euclidean}
    \[
    v_t(u_i) = \tilde v_t(u_i) -\proj_{\partial v(u_i)} (\tilde v_t(u_i))\]\[
    \text{with} \quad \tilde v_t(u_i) = - \Bigg[ \bigg(\ell'\bigg(\int \varphi \de \mu_{m,t}\bigg), \partial_j \varphi(u_i)\bigg) \Bigg]_{j=1}^d
    \]
\end{enumerate}
from \cite{santambrogio2017euclidean}, gradient flow selects subgradients of minimal norm

Observations:
\begin{itemize}
    \item velocity is the evaluation at each $u_i$ of the same vector field $v_t$
    \item given an initialization $u(0) \in \Omega^m$, this defines an atomic measure $\mu_{m,0}$ 
\end{itemize}
$\longrightarrow$ {\color{blue} makes sense to generalize to arbitrary measures $\mu_t$}
\end{frame}

\subsection{Generalize to infinte-dimensional gradient flow}
\begin{frame}{Generalize to Wasserstein gradient flow I}
Since 
\begin{enumerate}
    \item Evolution of $(\mu_t)_t$ under $(v_t)_t$ satisfies:
    \[ \partial_t\mu_t = -\mathrm{div}(v_t\mu_t)\] 
    \item Link between $v_t$ and $f$:\[v_t \in -\partial f'(\mu_{m,t})\]
        where \(
        f'(\mu)(u) := \bigg(\ell\bigg(\int\varphi\de\mu\bigg), \varphi(u)\bigg) + v(u)
        \)
\end{enumerate}
\bigskip 
\centering
{ \color{blue} $\Rightarrow$ we expect $(\mu_t)_{t}$ is a gradient flow on the space $\mP_2(\Omega)$: \it Wasserstein gradient flow}
\end{frame}

\begin{frame}{Generalize to Wasserstein gradient flow II}
\begin{definition}[Wasserstein gradient flow]
A Wasserstein gradient flow for the functional $f$ on a time interval $[0,T[$ is an absolutely continuous path $(\mu_t)_{t\in [0,T[}$ in $\mP_2(\Omega)$ that satisfies, distributionally on $[0,T[\times \Omega^d$,
\begin{equation*}\label{eq:gradientflow}
\partial_t \mu_t = - \mathrm{div} (v_t\mu_t) \quad \textit{ where } \,v_t \in - \partial f'(\mu_t) 
\end{equation*}
\end{definition}
\pause
{\color{blue} $\rightarrow$ It is well-defined starting from $\mu_0$ concentrated on a convex closed subset of $\Omega$} \cite{ambrosio2005gradient}

\pause 
\begin{block}{$W_2$ gradient flow generalizes particle}
    Whenever $(u_t)_t$ is a gradient flow for $f_m$, $t\mapsto \mu_{m,t} := \sum_{i=1}^m \delta_{u_i(t)}$ is a Wasserstein gradient flow for $f$!
\end{block}
\end{frame}

\begin{frame}{Many-particle limit}
    \begin{theorem}[Many particle limit]
    Consider $(t\mapsto u_{m}(t))_{m\in \mathbb{N}}$ a sequence of classical gradient flows for $f_m$ initialized in a closed convex set. If $\mu_{m,0}$ converges to some $\mu_0\in\mP_2(\Omega)$ for the Wasserstein distance $W_2$, then $(\mu_{m,t})_t$ converges, as $m\to \infty$, to the unique Wasserstein gradient flow of $f$ starting from $\mu_0$.
\end{theorem}
\bigskip
In a nutshell:
\begin{itemize}
    \item if $u(0)=(u_1(0),\ldots,u_m(0))$ are distributed according to $\mu_0$, then $\mu_{m,0}$ converges to $\mu_0$ by the law of large numbers
    \item {\color{blue} $\lim_{m\to\infty }(\mu_{m,t})_t = (\mu_t)_t$}
\end{itemize}
\end{frame}

\subsection{Global convergence guarantees}
% \begin{frame}{Final results}
% \begin{enumerate}
% \item The Wasserstein gradient flow exists and is unique for each starting point \cite{ambrosio2005gradient}
%     \item Under assumptions on $\varphi$ and the initialization, the limit of the particle gradient flow as $m \to \infty$ (if it exists) is the Wasserstein gradient flow
%     \item If Wasserstein gradient flow converges as $t\to\infty$, its limit is a minimizer for $f$
%     \item Then, if $(\mu_{m,t})_t$ is a gradient flow for $f_m$,
%     \[ 
%     \lim_{m,t\to\infty} f(\mu_{m,t})  = \min_{\mu_t\in\mP(\Omega)} f(\mu_t)
%     \]
% \end{enumerate}
% \end{frame}

\begin{frame}{Global convergence result}
    Structural assumptions:
    \begin{itemize}
        \item $\varphi, v$ are \textbf{2-homogeneous} ($\phi$ 1-homogeneous)
        \item the support of the initialization of the Wasserstein gradient flow satisfies a \textbf{``separation" property}: $B(0,r_b) \subset \mathbb{S}^{d-1}$ that separates $r_a\mathbb{S}^{d-1}$ and $r_b\mathbb{S}^{d-1}$ for $r_a<r_b$
    \end{itemize}
\begin{theorem}[Global convergence of particle gradient descent]
    Let $(\mu_t)_{t\geq 0}$ be a Wasserstein gradient flow of $f$ such that the support of $\mu_0$ is $S_0 \subset [-r_0,r_0]\times\Theta$ satisfies a separation property.
 If $(\mu_t)_t$ converges to $\mu_\infty$ in $W_2$, then $\mu_\infty$ is a global minimizer of $f$ over $\mP(\Omega)$. In particular, if $(u_m(t))_{m\in \N,t\geq 0}$ is a sequence of classical gradient flows initialized in $[-r_0,r_0]\times \Theta$ such that $\mu_{m,0}$ converges to $\mu_0$ in $W_2$ then
\[
\lim_{t,m\to \infty} f(\mu_{m,t}) = \min_{\mu \in \mP(\Omega)} f(\mu).
\]
\end{theorem}
\end{frame}

\begin{frame}{Application to ReLu neural networks}
Setting:
\begin{itemize}
    \item features live in $\R^{d-2}$, labels in $\R$
    \item $\ell$ is either the square or logistic loss 
\item 2-homogeneous case 
\item domain $\Theta$ is the disjoint union of 2 copies of $\R^d$
    \item $\varphi(p):x\mapsto \sigma(\sum_{i=1}^{d-1} s(p_i)x_i + s(p_{d})), \quad s(p_i)=p_i\vert p_i\vert$
    \item regularizer: $v(p)=\vert p\vert^2$ (as if $w=\vert p \vert$)

\end{itemize}
\begin{figure}
    \centering
    \includegraphics[width=0.9\linewidth]{Screenshot 2026-01-16 at 01.29.52.png}
    \caption{Training neural network with ReLU activation. Overfitting threshold $m=4$. Failure for $m=5$, success $m=10,100$.}
    \label{fig:placeholder}
\end{figure}
    
\end{frame}

\begin{frame}{Final remarks}

\begin{itemize}
    \item importance of initialization, as confirmed by extensive empirical literature
    \item particle gradient flow corresponds to continuous-time gradient descent: what can we say about discrete-step case? (double descent)
\end{itemize}
    
\end{frame}

\begin{frame}{Thank you!}
    \begin{figure}
        \centering
        \includegraphics[width=0.45\linewidth]{Screenshot 2026-01-15 at 18.55.36.png}
        \label{fig:placeholder}
    \end{figure}
\end{frame}

\section{References}
\begin{frame}[allowframebreaks]{References}
    \bibliographystyle{plain}
    \bibliography{refs}
\end{frame}

\begin{frame}{Recovering the lifting of (3)}
    \begin{enumerate}
        \item (equivalence of $\M$ and $\mP$ under homogeneity)
        \item surjectivity of $h_1$
        \item define $\varphi, g$ as above and prove equality of $f, j$
    \end{enumerate}
\end{frame}

\begin{frame}{Prerequisits}

 \begin{definition}[Subgradient and subdifferential] Given a function $f:\R^d\to \R$, the \emph{subgradient of $f$ at $x_0$}, $x_0\in\R^d$, is $p\in\R^d$ :
\[
f(x) \ge f(x_0) + p\cdot (x-x_0) + o(x-x_0) \qquad \forall x\in\R^d
\]
The set of all such $p$s is called the \emph{subdifferential} of $f$ at $x_0$ and we write $\partial f(x_0)$. The subdifferential is a closed and convex set.
\end{definition}

\begin{definition}[Gradient flow]
    A function $x:\R_+ \to \mathrm{Dom}(f)$ is a \emph{gradient flow} of $f:\mathrm{Dom}(f)=\R^d\to\R$ if $x$ is absolutely continuous over $\R_+$ and
    \[
    x'(t) \in -\partial f(x(t)) \qquad \text{for a.e. } t\in\R_+
    \]
\end{definition}
    
\end{frame}
\end{document}
