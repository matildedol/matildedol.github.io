\thispagestyle{plain}
\begin{center}
\large{\textbf{Abstract}}
\end{center}

The study of policy effects is core in empirical economic research. The effect of a policy is likely to vary across different units of a population. Recently, many approaches to estimate heterogeneous treatment effects have been proposed in econometrics, also touching upon the machine learning field. This has triggered the emergence of causal machine learning, and empirical economists have diverging opinion on whether it should be used. The main contribution of this study is to provide an exhaustive methodological review on the estimation of heterogeneous causal effects. Moreover, it investigates the impact of causal machine learning on econometrics, attempting to answer the question of whether and how it can be beneficial. It is found that the main methods to estimate heterogeneous effects are: including interactions in a linear regression; the marginal treatment effect estimator; matching on propensity scores; local regressions with differencing; double machine learning; and causal forests.  
Moreover, I argue that machine learning can be beneficial to econometrics, especially as it flexibly models data and can be adapted to accurately estimate causal effects, and therefore it should be integrated in this discipline. 